\documentclass{article}
\usepackage[utf8]{inputenc}
\usepackage{graphicx}
\usepackage{hyperref}
\usepackage{fancyhdr}
\usepackage[font=small,skip=0pt]{caption}
\pagestyle{fancy}

\fancyhead[LE,RO]{Jesse Both}
\fancyhead[LO,RE]{}


\hypersetup{
    colorlinks,
    citecolor=black,
    filecolor=black,
    linkcolor=black,
    urlcolor=black
}
\graphicspath{ {/graphics/} }
\title{\Huge{\textbf{CSE 321}  \\* Project 3 \\~\\ \textbf{Planning}}}

\date{} %remove date from make title


%image
% \begin{figure}[h!]
%   \begin{center}
%     \includegraphics[height=5cm]{graphics/().png}
%   \end{center}
%   \caption{CAPTION}
%   \label{fig:LABEL}
% \end{figure}

% Title Page
\begin{document}
%     \maketitle
%     \vfill 
%     {\Large\centering\textbf{Jesse Both  \\~\\}\par}

%     {\Large\centering{Fall 2021}\par}
%     {\large\centering{\today}\par}

%     \newpage
%     \begin{center}
%         \tableofcontents
%     \end{center}
% \newpage
\setcounter{secnumdepth}{-1}


\section{Problem Statement}
This system will provide additional senses to the hearing impaired.
some people are unable to use hearing aids so this would provide an 
alternative to hearing directly.  
\newline
\noindent
Much of our environmental awareness 
comes from our sense of hearing. This device give hearing impaired
individuals accessibility support by allowing them to be more aware
of their surroundings.  This device could prevent the user from 
getting themselves into a dangerous position or provide an indication
if someone nearby is trying to get their attention.
The system will provide a small vibration and give a visual que
when noise is sensed around them.

\section{Specifications}
\begin{itemize}
    \item System will detect sound.
    \item LEDs will visually display the sound level (green, yellow, red).
    \item Vibration motor will give feedback based on the sound level.
    \item Can adjust the sound sensitivity via a rotary encoder. Lowering sensitivity will 
            decrease feedback via the LEDs and vibration motor.
    \item Implementation must use:
    \begin{itemize}
        \item Watchdog
        \item Synchronization technique
        \item At least 1 bitwise control
        \item Critical section protection
        \item Task/Thread
        \item At least 1 Interrupt
    \end{itemize}
  \end{itemize}

\section{Constraints}
\begin{itemize}
    \item Detect sound and will stimulate an alternative sense rather than sound.
    \item Provide a visual and physical que that noise is nearby.
    \item Desired sound sensitivity can be adjusted.
    \item Toggle on and off the device.
  \end{itemize}

\section{Ask}
\begin{itemize}
    \item Purpose:
    \newline
    Provide an alternative hearing device to hearing aids.
    \item Inputs
    \begin{itemize}
        \item Audio Transducer  - Detects Sound
        \item Rotary Encoder - Adjusts the sound input levels
        \item Onboard Button - Toggle switch
    \end{itemize}

    \item Outputs
    \begin{itemize}
        \item LEDs - Visual sound indicator
        \item Vibration Motor - Physical sound indicator (may need more than 1)
    \end{itemize}
    \item Constraints
  \end{itemize}

\section{BOM}
\begin{itemize}
    \item LEDs - They light up.
    \begin{itemize}
        \item Red
        \item Yellow
        \item Green
    \end{itemize}
    \item Vibration Motor - Oscillating motor that causes vibrations.
    \item Audio Transducer - Detects sound as input.
    \item Rotary Encoder - Converts angular position to an input signal.
  \end{itemize}

\end{document}